\newpage
\begin{center}
  \textbf{\large АННОТАЦИЯ}
\end{center}

Наименование работы: автоматизированная платформа для развертывания и управления лямбда-функциями <<Lambda>>.

Цель работы: разработка платформы для автоматизированного развертывания и управления вычислительными задачами в контейнерах, обеспечивающей удобное взаимодействие пользователей с инфраструктурой и оптимизацию ресурсов.

Объект исследования: процессы автоматизированного развертывания, исполнения и мониторинга контейнеризированных вычислительных задач.

Предмет исследования: архитектурные, программные и алгоритмические решения, обеспечивающие эффективное управление контейнерами, балансировку нагрузки, обработку задач и анализ их выполнения в распределенной среде.

Объем работы составляет Х страниц. 
Работа включает в себя 3 главы, Х рисунков, Х таблиц, Х листингов кода.
Библиография включает Х источников.

Во введении раскрываются актуальность темы и практическая значимость разработки.

Первая глава посвящена анализу предметной области, формулировке целей и задач исследования, рассмотрению существующих аналогов.

Во второй главе пределяются функциональные возможности платформы и выибается технологический стек, описываются архитектура системы, пользовательский интерфейс и технические аспекты реализации.

Третья глава включает процесс тестирования, развертывания платформы, анализ её масштабируемости и возможных путей развития.

В заключении подводятся итоги работы, оценивается её практическая ценность и формулируются основные выводы.

\onehalfspacing
\setcounter{page}{6}

\newpage
\renewcommand{\contentsname}{\centerline{\large СОДЕРЖАНИЕ}}
\tableofcontents

\newpage
\begin{center}
  \textbf{\large ВВЕДЕНИЕ}
\end{center}
\addcontentsline{toc}{chapter}{ВВЕДЕНИЕ}

В современном мире информационных технологий все больше задач требует выполнения в автоматизированном и распределенном режиме.
Малые и средние компании, исследовательские группы и отдельные разработчики сталкиваются с необходимостью периодически запускать нерегулярные ресурсоемкие вычислительные процессы, такие как обработка данных, генерация отчетов, обучение моделей машинного обучения и других ресурсоемких операций.
Однако традиционные серверные решения требуют сложной настройки, постоянного администрирования и значительных финансовых вложений, что делает их использование не всегда целесообразным.

Существующие облачные платформы, большинство из которых не предсталены на Рооссийском рынке, такие как AWS Lambda, Google Cloud Functions и Azure Functions, предлагают serverless-вычисления, позволяя запускать код по требованию без необходимости управления серверами.
Однако эти решения ориентированы в первую очередь на крупный бизнес и имеют ограничения по конфигурации окружения, стоимости вызовов и способам взаимодействия с системой. Кроме того, использование этих сервисов в частных облаках или внутрикорпоративных инфраструктурах часто оказывается невозможным.
В качестве альтернативы некоторые компании развертывают собственные решения на базе Kubernetes, но это требует значительных инженерных ресурсов и опыта работы с контейнеризацией.

В данной работе рассматривается разработка платформы Lambda, предназначенной для автоматизированного развертывания и управления контейнеризированными вычислениями.
Платформа позволяет пользователям загружать и исполнять задачи в виде Docker-контейнеров с возможностью настройки параметров запуска, ограничений по ресурсам, управления окружением и интеграции с другими системами.

Предлагаемая платформа ориентирована на малые и средние компании, которым требуется удобное и экономически эффективное решение для выполнения вычислений без необходимости разворачивания и поддержки сложных серверных инфраструктур.

Таким образом, разработка платформы Lyambda является актуальной задачей, способной упростить доступ к автоматизированным вычислениям и снизить затраты на поддержку серверных мощностей.

