\onehalfspacing
\setcounter{page}{2}

\newpage
\renewcommand{\contentsname}{\centerline{\large СОДЕРЖАНИЕ}}
\tableofcontents

\newpage
\begin{center}
  \textbf{\large 1. ОБЩИЕ СВЕДЕНИЯ}
\end{center}
\refstepcounter{chapter}
\addcontentsline{toc}{chapter}{1. ОБЩИЕ СВЕДЕНИЯ}

\section{Наименование системы}

Автоматизированная платформа развертывания контейнеризованных \break функций в среде Kubernetes.

\section{Основания для проведения работ}

Приказ от XX.XX.2025 № XXXX-XX «О назначении руководителей и
утверждении тем выпускных квалификационных работ».

\section{Плановые сроки начала и окончания работы}

Плановый срок начала работ 01.03.2025.

Плановый срок окончания работ 01.06.2025.

\section{Источники и порядок финансирования}

Проект выполняется в рамках образовательного процесса на безвозмездной основе.

\section{Состав используемой нормативно-технической документации}

При разработке автоматизированной системы и создании проектно-
эксплуатационной документации Исполнитель должен руководствоваться
требованиями следующих нормативных документов:
\begin{itemize}
    \item[---]ГОСТ 19.201 Единая система программной документации. Техническое задание. Требования к содержанию и оформлению

    \item[---]ГОСТ Р 59792-2021 «Информационные технологии. Комплекс стандартов на автоматизированные системы. Виды испытаний автоматизированных систем».
    
    \item[---]ГОСТ Р 59853-2021 «Информационные технологии. Комплекс стандартов на автоматизированные системы. Автоматизированные системы. Термины и определения».
    
    \item[---]ГОСТ Р 59793-2021 "Информационные технологии. Комплекс стандартов на автоматизированные системы. Автоматизированные системы. Стадии создания"
    
    \item[---]ГОСТ Р 59795-2021 «Информационные технологии. Комплекс стандартов на автоматизированные системы. Автоматизированные системы.
    
    \item[---]ГОСТ 34.602-2020 «Информационные технологии. Комплекс стандартов на автоматизированные системы. Техническое задание на создание автоматизированной системы».
    
    \item[---]ГОСТ 34.201-2020 Межгосударственный стандарт. «Информационные технологии. Комплекс стандартов на автоматизированные системы. Виды, комплектность и обозначение документов при создании автоматизированных систем».
\end{itemize}

\newpage
\begin{center}
  \textbf{\large 2. НАЗНАЧЕНИЕ И ЦЕЛИ СОЗДАНИЯ СИСТЕМЫ}
\end{center}
\refstepcounter{chapter}
\addcontentsline{toc}{chapter}{2. НАЗНАЧЕНИЕ И ЦЕЛИ СОЗДАНИЯ СИСТЕМЫ}

\section{Назначение системы}

Основным назначением автоматизированной платформы развертывания контейнеризованных функций в среде Kubernetes является предоставление возможности пользователям и организациям развертывать и управлять контейнеризированными приложениями с минимальными усилиями, обеспечивая автоматизацию настройки инфраструктуры и поддерживая выполнение ресурсоемких задач.

\section{Цели создания системы}

Основной целью создания автоматизированной платформы развертывания контейнеризованных функций является упрощение и автоматизация процесса развертывания и управления контейнеризированными приложениями, что позволяет пользователям минимизировать усилия по настройке инфраструктуры и обеспечивать выполнение ресурсоемких вычислений.

Для реализации поставленной цели необходимо выполнить следующие задачи:

\begin{itemize}
\item[---]Провести анализ предметной области.
\item[---]Сравнить существующие аналогичные решения.
\item[---]Провести анализ целевой аудитории веб-приложения.
\item[---]Определить функциональные требования к веб-приложению.
\item[---]Разработать пользовательские сценарии.
\item[---]Спроектировать архитектуру веб-приложения.
\item[---]Разработать дизайн-макеты страниц икомпонентов веб-приложения.
\item[---]Спроектировать схему базы данных.
\item[---]Разработать серверную часть веб-приложения.
\item[---]Разработать клиентскую часть веб-приложения.
\item[---]Провести различные виды тестирования веб-приложения.
\end{itemize}

\newpage
\begin{center}
  \textbf{\large 3. ХАРАКТЕРИСТИКА ОБЪЕКТОВ АВТОМАТИЗАЦИИ}
\end{center}
\refstepcounter{chapter}
\addcontentsline{toc}{chapter}{3. ХАРАКТЕРИСТИКА ОБЪЕКТОВ АВТОМАТИЗАЦИИ}

\section{Объект автоматизации}

Объектом автоматизации является совокупность технологических процессов и операций, связанных с жизненным циклом контейнеризированных приложений.

\section{Существующее программное обеспечение}

На данный момент, выполнение задач, связанных с развертыванием и управлением контейнеризированными приложениями, осуществляется пользователями посредством прямого взаимодействия с компонентами базовой инфраструктуры.

Существующие подходы к размещению контейнеров включают:

\begin{itemize}
\item[---]размещение непосредственно на физических серверах;
\item[---]запуск в рамках виртуальных машин;
\item[---]использование платформы Docker для локального управления отдельными контейнерами;
\item[---]применение систем оркестрации типа Kubernetes для управления кластерами;
\item[---]использование сервисов, предоставляемых облачными провайдерами.
\end{itemize}

\newpage
\begin{center}
  \textbf{\large 4. ТРЕБОВАНИЯ К СИСТЕМЕ}
\end{center}
\refstepcounter{chapter}
\addcontentsline{toc}{chapter}{4. ТРЕБОВАНИЯ К СИСТЕМЕ}

\section{Требования к системе в целом}

\subsection{Требования к структуре и функционированию системы}

Разрабатываемая система включает веб-интерфейс для управления задачами и API для интеграции с внешними сервисами. Архитектура системы построена на сервис ориентированном, монолитном подходе, обеспечивая возможность быстрого перехода к микросервисной врхитектуре с ростом нагрузки. 

Основные модули системы включают:

\begin{itemize}
\item[---]модуль управления задачами;
\item[---]модуль мониторинга и логирования;
\item[---]модуль безопасности и авторизации;
\item[---]интерфейсный модуль.
\end{itemize}

Модуль управления задачами — отвечает за обработку запросов на запуск, остановку, и конфигурацию контейнеров. Поддерживает выполнение задач по расписанию, через API, или с помощью вебхуков.

Модуль мониторинга и логирования — собирает данные о состоянии контейнеров, ресурсах и предоставляет доступ к логам выполнения. 

Модуль безопасности и авторизации — управляет доступом к системе с использованием Keycloak для аутентификации и роли пользователей, а так же группами и досутпами пользователей в рамках системы.

Интерфейсный модуль — предоставляет удобный веб-интерфейс для взаимодействия пользователей с системой. Поддерживает просмотр задач, логов, и статистики в режиме реального времени, а также настройку параметров запуска контейнеров.

\subsection{Требования к взаимодействию и обмену информацией}

Система должна использовать RESTful API для взаимодействия между модулями и обеспечения передачи данных между компонентами платформы. Все данные передаются с использованием протокола HTTPS, что гарантирует защиту перехватки трафика.

Взаимодействие с системой возможно через веб-интерфейс и API, предоставляя пользователям гибкость в интеграции с другими сервисами. Для обработки запросов внешних систем поддерживается использование webhook.

\subsection{Перспективы развития и модернизации системы}

Система должна быть спроектирована с возможностью интеграции сторонних систем, гибкой адаптации функционала под потребности заказчика. Микросервисная архитектура позволяет расширять функционал и добавлять новые модули.

\subsection{Требования к численности и квалификации персонала}

Для начала эксплуатации системы необходим следующий минимальный состав персонала:

\begin{itemize}
\item[---]администратор системы — 2 шт. единицы;
\item[---]системный администратор — 1 шт. единица;
\item[---]руководитель отдела сопровождения — 1 шт. единица;
\item[---]инженеры DevOps — 2 шт. единицы;
\item[---]системные аналитики — 1 шт. единица;
\item[---]конечные пользователи (разработчики, инженеры, аналитики).
\end{itemize}

Администратор системы должен иметь высшее профессиональное образование в области информационных технологий (например, «Информационные системы и технологии» или «Программная инженерия»), а также практический опыт работы с системами виртуализации, оркестрации контейнеров (Kubernetes) и администрирования серверов.


Системный администратор должен иметь не менее среднего технического профессионального образования, должны обладать практическими навыками работы с серверным оборудованием.

Инженеры DevOps должен иметь не менее среднего технического профессионального образования, должны обладать практическими навыками работы с CI/CD-пайплайнами, контейнеризацией (Docker), настройкой систем мониторинга (Prometheus, Grafana) и управления нагрузкой.

Системный аналитик должен иметь высшее профессиональное образование в области информационных технологий (например, «Информационные системы и технологии» или «Программная инженерия»),  должен уметь описывать и анализировать бизнес-процессы, формировать техническую документацию и взаимодействовать с техническими специалистами для уточнения требований.

Руководитель отдела сопровождения должен иметь высшее профессиональное образование в области информационных технологий, должен обладать опытом управления проектами и знаниями архитектуры распределенных систем. 

Конечные пользователи должны владеть навыками работы с веб-приложениями и основами взаимодействия с REST API (для разработчиков). Обучение пользователей новым процессам и функционалу системы проводится в рамках внедрения платформы.

Квалификацию сотрудников контролирует и поддерживает сама организация, включая предоставление необходимых курсов повышения квалификации.

\subsection{Требования к показателям назначения}

В результате анализа на основе сценариев и проектирования с учетом масштабируемости, можно сделать вывод, что система должна обеспечивать работу одновременно не менее 1000 сотрудников с возможностью проведения опросов и анализа данных, обеспечивая при этом среднее время отклика не более 2 секунд на запросы пользователей.
Система должна быть способна обрабатывать не менее 5000 запросов в минуту, с возможностью параллельной обработки данных в режиме реального времени. Объем хранимых данных не должен превышать 5 ТБ на момент начала эксплуатации, с возможностью дальнейшего масштабирования.

\subsection{Требования к надежности}

Надежное функционирование системы должно быть обеспечено выполнением совокупности организационно-технических мероприятий, которые включают в себя:

\begin{itemize}
\item[---]организацию бесперебойного питания технических средств путем подключения сервера к источнику бесперебойного питания (ИБП);
\item[---]регулярное выполнение рекомендаций Министерства труда и социального развития РФ, изложенных в Постановлении от 23 июля 1998 года «Об утверждении межотраслевых типовых норм времени на работу по сервисному обслуживанию ПЭВМ и оргтехники и сопровождению технических средств».
\end{itemize}

\subsection{Требования к безопасности}

Все внешние элементы технических средств системы, находящиеся под напряжением, должны иметь защиту от случайного прикосновения, а сами технические средства иметь зануление или защитное заземление в соответствии с ГОСТ 12.1.030-81 и ПУЭ. Система электропитания должна обеспечивать защитное отключение при перегрузках и коротких замыканиях в цепях нагрузки, а также аварийное ручное отключение. Система должна быть защищена системой резервного питания для защиты от потери данных. Система резервного питания должна обеспечивать беспрерывную работу в течение 10 минут. Общие требования пожарной безопасности должны соответствовать нормам на бытовое электрооборудование. В случае возгорания не должно выделяться ядовитых газов и дымов. После отключения электропитания должно быть допустимо применение любых средств пожаротушения. Серверное помещение должно быть оборудовано автоматической системой пожаротушения и ручными огнетушителями (допустимого типа для тушения электроприборов). Факторы, оказывающие вредные воздействия на здоровье со стороны всех элементов системы (в том числе инфракрасное, ультрафиолетовое, рентгеновское и электромагнитное излучения, вибрация, шум, электростатические поля, ультразвук строчной частоты и т.д.), не должны превышать действующих норм (СанПиН 2.2.2./2.4.1340-03 от 03.06.2003 г.).

\subsection{Требования к эргономике и технической эстетике}

Проектирование пользовательского интерфейса системы надлежит выполнять согласно актуальным нормам эргономики, преследуя цель достижения высокого уровня юзабилити для различных групп пользователей. Расположение интерактивных компонентов и элементов управления следует оптимизировать для сокращения вероятности ошибочных действий пользователя и увеличения общей продуктивности взаимодействия с системой.

Взаимодействие пользователя с системой осуществляется посредством графических элементов интерфейса, таких как меню, управляющие кнопки и пиктограммы. Наряду с этим, должна быть обеспечена поддержка ввода команд с клавиатуры как альтернативного метода управления для пользователей, предпочитающих такой способ. Вся текстовая информация, представляемая в интерфейсе, включая системные сообщения, должна быть изложена на русском языке и ясно доносить необходимые сведения до пользователя. При фиксации ошибочных или нештатных действий пользователя, а также при системных сбоях, интерфейс обязан предоставлять пользователю ясные и содержательные уведомления о характере возникшей ситуации.

Необходимо гарантировать корректное отображение и функционирование интерфейса на широком спектре пользовательских устройств, включая смартфоны, планшетные компьютеры, персональные компьютеры и ноутбуки. Минимально поддерживаемое разрешение экрана устанавливается как 480x640 пикселей. Дизайн интерфейса должен реализовывать принципы адаптивности для корректной работы на экранах различных размеров, обеспечивая полноценный пользовательский опыт как на мобильных, так и на стационарных платформах. Все нетекстовые графические компоненты интерфейса следует снабжать альтернативным текстовым описанием (атрибут alt), предназначенным для вспомогательных технологий и пользователей с ограничениями по зрению, в полном соответствии с действующими стандартами веб-доступности.

Аппаратные средства, применяемые при создании и последующей эксплуатации системы, должны иметь сертификацию соответствия и удовлетворять нормам безопасности, установленным Росстандартом. Соблюдение этого требования является залогом стабильной и безопасной работы программно-аппаратного комплекса.

\subsection{Требования к эксплуатации, техническому обслуживанию, ремонту и хранению компонентов системы}

Система должна быть рассчитана на эксплуатацию в составе программно-технического комплекса Заказчика и учитывать разделение ИТ инфраструктуры Заказчика на внутреннюю и внешнюю. Техническая и физическая защита аппаратных компонентов системы, носителей данных, бесперебойное энергоснабжение, резервирование ресурсов, текущее обслуживание реализуется техническими и организационными средствами, предусмотренными в ИТ инфраструктуре Заказчика. 

Для нормальной эксплуатации разрабатываемой системы должно быть обеспечено бесперебойное питание ПЭВМ. При эксплуатации система должна быть обеспечена соответствующая стандартам хранения носителей и эксплуатации. ПЭВМ температура и влажность воздуха. Заказчик обязан контролировать техническое состояние оборудования, в случае технических неисправностей – проводить своевременное техническое обслуживание, а также проводить регулярное техническое обслуживание, обеспечивать постоянную чистоту серверных помещений, а также обеспечивать выполнение всех условий по эксплуатации, предоставленные заводом-изготовителем. Исполнитель не несёт ответственности за ущерб, полученный в ходе действия и/или бездействия заказчика при проведении технического обслуживания и обеспечения условий эксплуатации. 

Периодическое техническое обслуживание технических средств должны включать в себя обслуживание всех используемых средств, включая рабочие станции, серверы, кабельные системы и сетевое оборудование, устройства бесперебойного питания. В процессе проведения периодического технического обслуживания должны проводиться внешний и внутренний осмотр и чистка технических средств, проверка контактных соединений, проверка параметров настроек работоспособности технических средств и тестирование их взаимодействия. Восстановление работоспособности технических средств должно проводиться в соответствии с инструкциями разработчика и поставщика технических средств и документами по восстановлению работоспособности технических средств и завершаться проведением их тестирования. 

Размещение помещений и их оборудование должны исключать возможность бесконтрольного проникновения в них посторонних лиц и обеспечивать сохранность находящихся в этих помещениях конфиденциальных документов и технических средств. Размещение оборудования, технических средств должно соответствовать требованиям техники безопасности, санитарным нормам и требованиям пожарной безопасности. 

Все пользователи системы должны соблюдать правила эксплуатации электронной вычислительной техники. Квалификация персонала и его подготовка должны соответствовать технической документации.

\subsection{Требования к защите информации от несанкционированного доступа}

Система должна обеспечивать защиту от несанкционированного доступа (НСД) на уровне, не ниже установленного требованиями, предъявляемыми к категории 1Д по классификации действующего руководящего документа Гостехкомиссии России «Автоматизированные системы. Защита от несанкционированного доступа к информации. Классификация автоматизированных систем» 1992 г. 

Компоненты подсистемы защиты от НСД должны обеспечивать:

\begin{itemize}
\item[---]идентификацию пользователя;
\item[---]проверку полномочий пользователя при работе с системой;
\item[---]разграничение доступа пользователей на уровне задач и информационных массивов.
\end{itemize}

Протоколы аудита системы и приложений должны быть защищены от несанкционированного доступа как локально, так и в архиве. Уровень защищенности от несанкционированного доступа средств вычислительной техники, обрабатывающих конфиденциальную информацию, должен соответствовать требованиям к классу защищенности 6 согласно требованиям действующего руководящего документа Гостехкомиссии России «Средства вычислительной техники.

Защита от несанкционированного доступа к информации. Показатели защищенности от несанкционированного доступа к информации». Защищённая часть системы должна использовать «слепые» пароли (при наборе пароля его символы не показываются на экране либо заменяются одним типом символов, количество символов не соответствует длине пароля).

Защищённая часть системы должна автоматически блокировать сессии пользователей и приложений по заранее заданным временам отсутствия активности со стороны пользователей и приложений. Защищённая часть системы должна использовать многоуровневую систему защиты. Защищённая часть системы должна быть отделена от незащищённой части системы межсетевым экраном.

\subsection{Требования по сохранности информации при авариях}

Программное обеспечение должно восстанавливать свое функционирование при корректном перезапуске аппаратных средств. Должна быть предусмотрена возможность организации автоматического и ручного резервного копирования данных системы средствами системного и базового программного обеспечения (ОС, СУБД), входящего в состав программно-технического комплекса Заказчика. Приведенные выше требования не распространяются на компоненты системы, разработанные третьими сторонами и действительны только при соблюдении правил эксплуатации этих компонентов, включая своевременную установку обновлений, рекомендованных производителями покупного программного обеспечения.

\subsection{Требования по патентной чистоте}

Установка системы в целом, как и установка отдельных частей системы не должна предъявлять дополнительных требований к покупке лицензий на программное обеспечение сторонних производителей, кроме лицензионных версий ПО, указанного в дополнительных соглашениях.

\section{Требования к функциям (задачам), выполняемым системой}

Существующие подсистемы в ходе модернизации не должны потерять свой текущий функционал. Все подсистемы должны обеспечивать работу в рамках одной авторизации пользователя за сеанс, не допускается повторный запрос авторизации при переходе в другую подсистему.

Данные о текущем сеансе должны передаваться в соответствующие подсистемы автоматическим способом. Все подсистемы должны дополнять функционал друг друга. Не допускается предоставление противоречивых данных между подсистемами.

\subsection{Пользовательские роли и права}

В разрабатываемой системе, на этапе прототипирования, предусмотрен один тип пользоватей - пользователь платформы.

Пользователь может создавать, просматривать и управлять задачами в своем пространстве и в рамках групп пользователей в которых он состоит.

Незарегистрированным и неавторизированным пользователям предоставляется возможность зарегистривароться или авторизоваться на платформе. После успешной авторизации каждый пользователь получает доступ к полному функционалу.

\subsection{Подсистема регистрации, авторизации, управления групповым доступом}

Подсистема предназначена для реализации функций управления идентификацией пользователей и контроля доступа к ресурсам платформы. Она обеспечивает процессы аутентификации, авторизации, управления учетными записями и организации пользователей в группы для совместной работы.

Подсистема должна обеспечивать функциональность для регистрации новых пользователей в системе. Для зарегистрированных пользователей должна быть реализована процедура аутентификации на основе предоставленных учетных данных. Необходимо предусмотреть безопасный механизм восстановления доступа к учетной записи в случае утраты пользователем пароля.

В результате успешной аутентификации подсистема обязана генерировать и предоставлять пользователю JWT токены доступа, соответствующие стандартам OAuth 2.0. 

Подсистема, а также интегрированные с ней компоненты, должны реализовывать строгую валидацию предъявляемых токенов, включая проверку криптографической подписи, срока действия и эмитента.

Подсистема должна реализовывать модель управления доступом на основе ролей (Role-Based Access Control - RBAC). Необходимо обеспечить возможность создания пользователями логических групп для организации совместного доступа к задачам. Пользователь, инициировавший создание группы, по умолчанию должен становиться администратором созданной группы.

Должен быть реализован механизм динамического создания набора ролей, специфичных для каждой группы, соответствующих предопределенным уровням разрешений, таким как: право на просмотр (VIEW), право на модификацию (EDIT), право на инициирование выполнения (RUN) и административные права (ADMIN) в контексте ресурсов группы.

Должна быть обеспечена возможность делегирования этих ролей участникам группы администратором группы.

\subsection{Подсистема развертывания и управления задачами}

Подсистема должна позволять пользователям создавать новые задачи через API. При создании нужно указывать настройки: имя задачи, кто ее создал (пользователь или группа), какой Docker-образ использовать, какую команду запустить в контейнере, какие переменные окружения задать. Также нужно указать, как запускать задачу: сразу, по расписанию (cron) или по внешнему сигналу (webhook). Для задач по расписанию нужно сохранить само расписание. Все эти настройки задачи должны сохраняться в базе данных. Пользователи также должны иметь возможность удалять задачи, чтобы они больше не запускались.

Запуск задач происходит в зависимости от их типа. Задачи могут запускаться немедленно по команде пользователя, автоматически по заданному расписанию или при получении специального запроса (webhook). Для фактического запуска подсистема обращается к Kubernetes. Она создает объекты Job для разовых запусков или CronJob для задач по расписанию. В Kubernetes передаются все нужные настройки контейнера (образ, команда, переменные). Пользователи должны иметь возможность через API остановить уже запущенную задачу или запустить ее снова.

Подсистема должна следить за тем, как выполняются задачи в Kubernetes. Она должна записывать время начала и окончания каждого запуска. Когда задача завершается, подсистема должна получить из Kubernetes ее итоговый статус, код завершения и логи. Вся информация запуске (статус, время, логи, ошибки) должна сохраняться в базе данных и быть связана с исходной задачей.

Пользователи должны иметь возможность получить список всех доступных им задач. Также они должны мочь посмотреть подробную информацию о задаче, включая ее настройки и всю историю ее запусков с результатами.

\section{Требования к видам обеспечения}

\subsection{Требования к математическому обеспечению}

Математические методы и алгоритмы, используемые для шифрования/дешифрования данных, обработке и систематизации полученных результатов опросов, а также программное обеспечение, реализующее их, должны быть сертифицированы уполномоченными организациями для использования в государственных органах Российской Федерации.

\subsection{Требования к информационному обеспечению системы}

Состав, структура и способы организации данных в системе должны быть определены на этапе технического проектирования. Хранение данных должно осуществляться на основе современных реляционных СУБД. Для обеспечения целостности данных должны использоваться встроенные механизмы СУБД. 

Средства СУБД, а также средства используемых операционных систем должны обеспечивать документирование и протоколирование обрабатываемой в системе информации. Структура базы данных должна поддерживать кодирование хранимой и обрабатываемой информации в соответствии с общероссийскими классификаторами (там, где они применимы). 

Доступ к данным должен быть предоставлен только авторизованным пользователям с учетом их служебных полномочий, а также с учетом категории запрашиваемой информации. Структура базы данных должна быть организована рациональным способом, исключающим единовременную полную выгрузку информации, содержащейся в базе данных системы. Технические средства, обеспечивающие хранение информации, должны использовать современные технологии, позволяющие обеспечить повышенную надежность хранения данных и оперативную замену оборудования (распределенная избыточная запись/считывание данных, зеркалирование, независимые дисковые массивы, кластеризация). 

\subsection{Требования к лингвистическому обеспечению системы}

Всё прикладное программное обеспечение системы для организации взаимодействия с пользователем должно использовать английский язык.

\subsection{Требования к программному обеспечению системы}

При проектировании и разработке системы необходимо эффективным образом использовать ранее закупленное программное обеспечение, как серверное, так и для рабочих станций. Базовой программной платформой должна являться операционная система ОС Ubuntu Server 22.04 LTS. 

\subsection{Требования к техническому обеспечению}

Техническое обеспечение системы должно базироваться на максимально эффективном использовании существующих вычислительных и сетевых ресурсов, предоставленных Заказчиком.

Предполагается, что архитектура технического обеспечения допускает возможность дальнейшего масштабирования (увеличения количества серверов, расширения сетевой инфраструктуры) по мере роста требований к производительности и объему обрабатываемых данных платформы.

Используемое серверное обеспечение должно поддерживать технологии виртуализации для возможности создания кластера Kubernetes.

\subsection{Требования к организационному обеспечению}

Организационное обеспечение системы должно быть достаточным для эффективного выполнения персоналом возложенных на него обязанностей при осуществлении автоматизированных и связанных с ними неавтоматизированных функций системы. Заказчиком должны быть определены должностные лица, ответственные за предоставление работникам следующих пользовательских ролей в системе.

\newpage
\begin{center}
  \textbf{\large 5. ПОРЯДОК РАЗРАБОТКИ АВТОМАТИЗИРОВАННОЙ СИСТЕМЫ}
\end{center}
\refstepcounter{chapter}
\addcontentsline{toc}{chapter}{5. ПОРЯДОК РАЗРАБОТКИ АВТОМАТИЗИРОВАННОЙ СИСТЕМЫ}

\section{Порядок организации разработки АС}

Этапами организации разработки системы должны являться:

\begin{enumerate}
\item прототипирование;
\item создание дизайна;
\item разработка;
\item тестирование;
\item документирование;
\item обучение персонала.
\end{enumerate}

\section{Перечень документов, предъявляемых по окончании}

Соответственно разделу 6.1 по окончании этапа соответственно
должны быть предъявлены:

\begin{enumerate}
\item рабочий прототип, который будет отражать основные функции ивозможности сайта;
\item дизайн проекта;
\item полнофункциональная версия сайта;
\item работоспособный сайт, прошедший испытания и готовый кэксплуатации;
\item программная и эксплуатационная документация.
\end{enumerate}

\section{Порядок проведения экспертизы технической документации}

Экспертиза технической документации должна проводиться в
соответствии с ГОСТ Р 50.03.01-2017.

\section{Порядок разработки, согласования и утверждения плана}

Согласование и утверждение плана совместных работ осуществляется
Заказчиком и Исполнителем. Согласованный и утвержденный план
совместных работ по созданию системы является обязательным для всех
участников. В процессе работ Исполнитель по согласованию с Заказчиком
может уточнять и корректировать план совместных работ в пределах условий
настоящего Технического задания.

\section{Требования к гарантийным обязательствам разработчика}

Качество оказания услуг должно быть в соответствии с требованиями
Технического задания. Оказание услуг должно осуществляться с соблюдением
трудового законодательства Российской Федерации в части обеспечения
требований по нормам выработки, режиму работы и условиям отдыха.

\section{Порядок разработки, согласования и утверждения программы}

Порядок разработки согласования и утверждения программы
метрологического обеспечения не определяется.

Порядок разработки согласования и утверждения программы
эргономического обеспечения не определяется.

Состав, содержание и последовательность работ по программе
обеспечения надежности системы определяет Исполнитель.

Исполнитель должен проводить оценку надежности системы на этапе
разработки системы.

\newpage
\begin{center}
  \textbf{\large 6. ПОРЯДОК КОНТРОЛЯ И ПРИЕМКИ СИСТЕМЫ}
\end{center}
\refstepcounter{chapter}
\addcontentsline{toc}{chapter}{6. ПОРЯДОК КОНТРОЛЯ И ПРИЕМКИ СИСТЕМЫ}

\section{Порядок контроля и приемки системы}

Порядок контроля и приёмки системы приведён ниже:

\begin{enumerate}
\item проверка работоспособности системы;
\item проверка сопутствующих документов;
\item определение комплектности системы.
\end{enumerate}

Приемно-сдаточные испытания должны проводиться согласно этапу
тестирования.

\section{Общие требования к приемке работ}

Ход проведения приемно-сдаточных испытаний документируется в
протоколе проведения испытаний. После проведения испытаний в полном
объеме, на основании протокола испытаний утверждают свидетельство о
приемке и подписывают акт сдачи-приёма работы.

\section{Статус приёмной комиссии}

Статус приёмной комиссии определяется заказчиком до проведения
испытаний.

\newpage
\begin{center}
  \textbf{\large 7. ТРЕБОВАНИЯ К СОСТАВУ И СОДЕРЖАНИЮ РАБОТ ПО ПОДГОТОВКЕ ОБЪЕКТА АВТОМАТИЗАЦИИ К ВВОДУ СИСТЕМЫ В ДЕЙСТВИЕ}
\end{center}
\refstepcounter{chapter}
\addcontentsline{toc}{chapter}{7. ТРЕБОВАНИЯ К СОСТАВУ И СОДЕРЖАНИЮ РАБОТ ПО ПОДГОТОВКЕ ОБЪЕКТА АВТОМАТИЗАЦИИ К ВВОДУ СИСТЕМЫ В ДЕЙСТВИЕ}

В ходе выполнения проекта на объекте автоматизации требуется
выполнить работы по подготовке к вводу системы в действие. При подготовке
к вводу в действие АС заказчик должен обеспечить выполнение следующих
работ:

\begin{enumerate}
\item определить подразделение и ответственных должностных лиц, ответственных за внедрение и проведение опытной эксплуатации;
\item обеспечить присутствие пользователей на обучении работе с системой;
\item обеспечить соответствие помещений и рабочих мест пользователей системы в соответствии с требованиями, изложенными в настоящем ТЗ;
\item обеспечить выполнение требований, предъявляемых к программно-техническим средствам, на которых должно быть развернуто программное обеспечение;
\item совместно с исполнителем подготовить план развертывания системы на технических средствах заказчика;
\item провести опытную эксплуатацию системы.
\end{enumerate}

\newpage
\begin{center}
  \textbf{\large 8. ТРЕБОВАНИЯ К ДОКУМЕНТИРОВАНИЮ}
\end{center}
\refstepcounter{chapter}
\addcontentsline{toc}{chapter}{8. ТРЕБОВАНИЯ К ДОКУМЕНТИРОВАНИЮ}

Состав программной документации:

\begin{enumerate}
\item техническое задание;
\item эскизный проект;
\item технический проект;
\item пояснительная записка к техническому проекту;
\item текст программы;
\item программа и методика испытаний;
\item технические условия;
\item руководство программиста;
\item руководство пользователя;
\item руководство системного администратора;
\item руководство администратора баз данных.
\end{enumerate}

\newpage
\begin{center}
  \textbf{\large 9. СПИСОК ИСПОЛЬЗОВАННЫХ ИСТОЧНИКОВ}
\end{center}
\refstepcounter{chapter}
\addcontentsline{toc}{chapter}{9. СПИСОК ИСПОЛЬЗОВАННЫХ ИСТОЧНИКОВ}

Настоящее Техническое задание разработано на основе следующих
документов и информационных материалов:

\begin{enumerate}
\item ГОСТ Р 59795-2021. Комплекс стандартов на автоматизированные системы. Автоматизированные системы. Стадии создания. URL: $https://www.astoni.ru/upload/iblock/2d4/GOST-34.601\_90.pdf$ (дата обращения 2025.04.24).
\item ГОСТ 59793-2021. Информационная технология. Комплекс стандартов на автоматизированные системы. Виды, комплексность и обозначение документов при создании автоматизированных систем. URL: $https://nd-gsi.ru/ntd/gost/gost\_34.201-89.pdf$ (дата обращения 2025.04.24).
\item ГОСТ 59794-2021. Методические указания. Информационная технология. Комплекс стандартов на автоматизированные системы. Автоматизированные системы. Требования к содержанию документов. URL: $http://a-podkidyshev.ru/GOST/RD-50-34-698-90-AS.pdf$ (дата обращения 2025.04.24).
\item ГОСТ 7.32-2017. Система стандартов по информации, библиотечному и издательскому делу. Отчёт о научно-исследовательской работе. Структура и правила оформления. URL: $https://www.rea.ru/ru/org/managements/orgnirupr/Documents/gost_7.32-2017.pdf$ (дата обращения 2025.04.24).
\item СанПиН 2.2.2./2.4.1340-20 Гигиенические требования к персональным электронно-вычислительным машинам и организации работы от 03.06.2020 г. URL: $https://stavsch36.ru/doc/post_306_2003.pdf$ (дата обращения 2025.04.24).
\item СанПин 2.2.3670-21 Санитарно-эпидемиологические требования к условиям труда от 01.01.2021. URL: $https://docs.cntd.ru/document/573230583$ (дата обращения 2025.04.24).
\item ГОСТР 14915-1-2016. Эргономика мультимедийных пользовательских интерфейсов. URL: $https://files.stroyinf.ru/Data/631/63124.pdf$ (дата обращения 2025.04.24).
\end{enumerate}
