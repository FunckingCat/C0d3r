\newpage
\begin{center}
  \textbf{\large 1. ПРЕДМЕТНАЯ ОБЛАСТЬ}
\end{center}
\refstepcounter{chapter}
\addcontentsline{toc}{chapter}{1. ПРЕДМЕТНАЯ ОБЛАСТЬ}

\section{Предметная область разработки бессерверных приложений}

Бессерверные архитектуры представляют собой облачную модель исполнения программного кода, при которой используется модель запуска серверных вычислений <<Backend as a service>> (BaaS). \cite{roberts2017serverless}

Бессерверность - это общая технология, позволяющая не связываться с базовой инфраструктурой. Двумя областями применения бессерверной технологии являются серверная часть как услуга (BaaS) и Функция как услуга (FaaS) или серверная часть как услуга и функция как услуга соответственно. 

BaaS (Серверная часть как услуга) — это облачное решение, предоставляющее готовую инфраструктуру для серверных приложений. Оно включает в себя набор сервисов, таких как аутентификация пользователей, база данных, файловое хранилище, push-уведомления и другие стандартные серверные компоненты. BaaS позволяет разработчикам сосредоточиться на логике приложения, не беспокоясь о развертывании и обслуживании серверной части. Примеры BaaS: Firebase, Parse, Back4App.

FaaS (Функция как услуга) - это вычислительная модель, в которой разработчик записывает отдельные функции, которые выполняются по запросу (например, при запуске события). Эти функции выполняются в ответ на такие события, как HTTP-запросы или изменения в базе данных, и масштабируются автоматически. В отличие от BaaS, который предоставляет более широкий спектр серверных сервисов, FaaS фокусируется исключительно на выполнении кода. Примеры FaaS: AWS Lambda, облачные функции Google, функции Azure.\cite{elsten2023exploring}

Применение бессерверных вычислений, в отличие от традиционных серверных, позволяет не заботиться о даступности и масшатбировании инфораструктуры, а так же снижать издержки возникающие из-за простоев оборудования в периоды малой нагрузки.\cite{pu2019shuffling}

Несмотря явное указание отстуствия серверов при применении бессерверных вычилений, сервера при использовании данного подхода импользуются, но управление ими передается облачному провайдеру.

Изначально термин <<Serverless>> использовался в контексте rich-client\cite{lowber2001thin} приложений,  одностраничные веб-приложения (SPA)\cite{kokkonen2015single} или мобильных приложений, отличительной чертой которых является использование обширной экосистемы доступных в облаке данных, таких как удаленные базы данных (например Firebase), службы аутентификации и так далее. 

К serverless приложениям также относятся приложения, в которых серверная логика по-прежнему реализуется разработчиком приложения, но, в отличие от традиционных архитектур, она выполняется в вычислительных контейнерах без сохранения состояния, которые запускаются по событиям. Такие контейнеры недолговечны, полностью управляются третьей стороной, так же существуют только в рамках одного вызова.

Один из способов реализовать такой подход это - “Function as a service” или “FaaS”. AWS Lambda в настоящее время является одной из самых популярных реализаций платформы "Function as a service", но существует также множество других.

Несмотря на все преимущества которые дает применение бессерверных вычилений, необходимо учитывать особенности разработки под бессерверные системы и знать о налогах которые накладывает применение бессерверных вычислений.

Наиболее важной может оказаться пролема холодного запуска\cite{liu2023faaslight}. При масшатбировании до 0 при получении запроса на выполнение функции значительное время может быть затрачено на первоначальный запуск конейнера с необходимым сервисом\cite{silva2020prebaking}. 

\section{Анализ аналогов и конкурентов}

Первое появление концепции бессерверных вычислений связано с AWS Lambda, запущенной в 2014 году Amazon Web Services. Это решение позволило разработчикам загружать небольшие функции, которые выполнялись в ответ на такие события, как HTTP-запросы, изменения в базе данных или сообщения в очереди.

После успеха AWS Lambda, другие крупные облачные провайдеры представили свои serverless-решения:
\begin{itemize}
    \item[---] 2016г. — Google Cloud Functions;
    \item[---] 2016г. — Microsoft Azure Functions;
    \item[---] 2017г. — IBM Cloud Functions;
    \item[---] 2019г. — Google Cloud Run.
\end{itemize}

Ни один из крупнейших поставщиков Serverless решений на данный момент не доступен в России.

В России на данной момент доступны поставщики serverless инфраструктуры:
\begin{itemize}
    \item[---] Yandex Cloud;
    \item[---] Cloud.ru
\end{itemize}

С развитием контейнерных технологий появилась возможность объединять Serverless и Container-based подходы, что позволило избежать жестких ограничений FaaS.

Для возможности плавной смены поставщика serverless инфораструктуры существуют open-source фреймворки позволяющие разрабатывать решения одинаково работающие на инфраструктуре различных поставщиков, реализующих поддержку данных решений. К таким решениям можно отнести:
\begin{itemize}
    \item[---]Apache Openwhisk https://openwhisk.apache.org/
    \item[---]Fission https://fission.io/
    \item[---]Fn https://fnproject.io/
    \item[---]Knative https://knative.dev/docs/
    \item[---]Kubeless https://github.com/vmware-archive/kubeless
    \item[---]Nuclio https://nuclio.io/
    \item[---]OpenFaaS https://www.OpenFaaS.com/
\end{itemize}

Рассмотрим основных игроков на рынке serverless решений.

\subsection{AWS Lambda}

AWS Lambda является флагманом подхода <<Function as a Сode>>, одним из первых и самых популярных решений для запуска облачных функций. Сервисы работает не с контейнерами, а напрямую с програмным кодом, что накладывает ограничения на стек технологий которые могут быть использованы при интеграции с данным сервисом.

К плюсам системы AWS Lambda можно отнести:
\begin{itemize}
    \item[---]Полная интеграция с AWS-экосистемой (S3, DynamoDB, API Gateway).
    \item[---]Автоматическое масштабирование.
    \item[---]Поддержка множества языков (Python, Node.js, Java, Go и др.).
    \item[---]Гибкая система триггеров (HTTP, очередь сообщений, события облака).
\end{itemize}

Минусами системы являются:
\begin{itemize}
    \item[---]Ограничение времени выполнения (15 минут).
    \item[---]Задержки из-за холодных стартов.
    \item[---]Высокая стоимость при интенсивных нагрузках.
    \item[---]Недоступность в России.
\end{itemize}

\subsection{Google Cloud Run}

Google Cloud Run, продукт разарботанный компанией Google, является представителем класса BaaS систем которые используют в качестве единицы развертывания контейнер.
В отличии от систем использующих в качестве единицы развертывания код, подход с конетейнерами не накладывает никаких ограничений на стек используемых технологий и сложность разворачиваемого сервиса. Единственным требованием к запускамому юниту является соответствие спецификации Open Container Initiative (OCI)\cite{initiativeopen}.

К плюсам системы Google Cloud Run можно отнести:
\begin{itemize}
    \item[---]Поддержка любых языков и сред (т.к. работает с контейнерами).
    \item[---]Простота развертывания (автоматически масштабируемые контейнеры).
    \item[---]Гибкость в выборе окружения.
\end{itemize}

Минусами системы являются:
\begin{itemize}
    \item[---]Более высокая цена по сравнению с FaaS-решениями.
    \item[---]Холодные старты при отсутствии постоянной нагрузки.
    \item[---]Сильная интеграция с Google Cloud (ограниченная независимость).
    \item[---]Недоступность в России.
\end{itemize}

\subsection{OpenFaaS, OpenWhisk, Knative и т.д.}

OpenFaaS, OpenWhisk, Knative являются популярными open-source\cite{bretthauer2001open} решениями для развертиывания локальных FaaS сервисов. В отличие от проприетарных облачных FaaS-платформ такие решения, при использовании для управления локальной инраструктурой позволяют избежать ограничений на время выполнения и более точно пронозировать и управлять нагрузкой. Некоторые из представленных на рынке систем могут развернуты на любых можностях: Kubernetes, Docker Swarm, локальные сервера. Все представленные сисемы работаю с контейнерами как с единицей развертывания, а значит так же накладывают ограничение соответствия стандарту OCI.

К плюсам open-source систем разворачиваемым локально можно отнести:
\begin{itemize}
    \item[---] Полный контроль над инфраструктурой.
    \item[---] Оптимизация и прогнозируемость затрат.
    \item[---] Технологическая независимость.
\end{itemize}
    
Минусами систем являются:
\begin{itemize}
    \item[---] Необходимость наличия команды поддержки и технических компетенций.
    \item[---] Потребность в ресурсах.
    \item[---] Необходимость самостоятельной настройки сервисов.
    \item[---] Отсутствие поддержки вендора.
\end{itemize}

\subsection{Аналоги}

Аналогом же разрабатываемого решения будет являться использования классических решений размещения приложений на собственных или арендрованных мощностях, таких как виртульные машины, выделенные серверы, управляемые Kubernetes или Docker Swarm кластеры.

Использование традиционных решений для размещения приложений на собственных или арендованных мощностях, таких как виртуальные машины (VM), выделенные серверы, а также управляемые кластеры Kubernetes или Docker Swarm, имеет свои преимущества и недостатки.

Преимущества:
\begin{itemize}
    \item[---]Полный контроль над инфраструктурой.
    \item[---]Гибкость в выборе технологий.
    \item[---]Предсказуемость затрат.
\end{itemize}

Недостатки:
\begin{itemize}
    \item[---]Необходимость наличия команды поддержки и технических компетенций.
    \item[---]Потребность в ресурсах.
    \item[---]Необходимость самостоятельной настройки сервисов.
    \item[---]Высокие капитальные затраты при простоях вычислительных мощностей.
    \item[---]Сложности с масштабированием.
    \item[---]Ограниченная гибкость.
\end{itemize}

\subsection{Гибридный подход к управлению инфраструктурой}

В следствие наличия плюсов и минусов у каждого подхода к управлению инфраструктурой, большинству компаний подойдет гибридный подход, объединяющий локальные ресурсы с облачными сервисами, позволяет достичь баланса между контролем над инфраструктурой, гибкостью в масштабировании и эффективным управлением затратами.

При исопльзовании гибридного подхода сервисы испытывающие постоянную нагрузку, а так же сервисы время отклика которых критично для операций которые они выполняют не должны быть развернуты с использованием классических подходов усправления доставкой и развертыванием.
В это же время использование FaaS сервисов для функционала не не находящегося под постоянной нагрузкой и не имеющего строгих трбований к времени ответа, может помоч оптимизировать расходы на серверную инфораструктуру и простои оборудоания.

\section{Анализ целевой аудитории}

\subsection{Сегментация целевой аудитории}

На начальных этапах разработки платформы целевой аудиторией разрабатываемой системы являются стартапы, разработчики, малые и средние кампании, а так же образовательные учереждения, у которых есть потребность в выполнении задач, лежащих в прикладной области платформы.

Важным аспектом целевой аудитории является отсутствие собсевнной сервеной инфраструктуры или отсутствие возможности избежать ее простоев. Целевая аудитория системы не имеет возможности и компетенций что бы использовать существующие open-source решения, такие как OpenFaaS, OpenWhisk, Knative на мощностях собсевнной инфраструктуры.

Для целевой аудитории целью использования платформы является автоматизация и отпимизация процессов компании, выполнение требователных вычислений, лежащих в области деятельности комании.

\subsection{Анализ потребностей}

Исходя из выявленных сегментов целевой аудитории, ключевые потрбности можно условно разделить на несколько направлений:
\begin{itemize}
    \item[---]компенсания отстуствия собсевнной инфораструктуры;
    \item[---]оптимизация затрат на развертывание и эксплуатацию;
    \item[---]выполнение ресурсоемких вычислений.
\end{itemize}

\subsection{Выводы и рекомендации для разработки}

На основании проведенного анализа целевой аудитории, можно сделать следующие выводы:

\begin{itemize}
    \item[---]предавители сегментов целевой аудитории нуждаются в локализованном, легком в управлении решениии для запуска бессерверных вычислений, которое позволит избежать зависимости от зарубежных вендоров;
    \item[---]существующие open-source решения не являются прямыми конкурентами, но поддержка спецификаций, позволяющих совместимость с другими существующими решениями будет конкурентным преимуществом;
\end{itemize}

\section{Проблема, цель и задачи разработки}

Современные решения задач в области выполнения бессерверных вычислений в настоящее время сопряжены с радом проблем и компромисов, таких как недоступность крупных FaaS сервисов в России, органичения по формату, времени выполнения, нагрузке выполняемых функций. 

Разрабаотываемая плаотаорма автоматизации управления бессерверными вычислениями призвана устранить часть этих барьеров, предоставив удобное и эффективное решение для выполнения задач выполнения бессерверных вычислений различных типов.

Целью разработки является создание платформы управления бессерверными вычислениями, обеспечивающую балланс между простотой испольования, гибкостью конфигурации и предстказуемостью затрат.

Для достижения целей разработки необходимо выполнить следующие задачи:

\begin{itemize}
    \item[---]провести анализ предметной области;
    \item[---]сравнить существующие аналогичные решения;
    \item[---]провести анализ целевой аудитории веб-приложения;
    \item[---]определить функциональные требования к веб-приложению;
    \item[---]разработать пользовательские сценарии;
    \item[---]спроектировать архитектуру веб-приложения;
    \item[---]разработать дизайн-макеты страниц и компонентов веб-приложения;
    \item[---]спроектировать схему базы данных;
    \item[---]разработать серверную часть веб-приложения;
    \item[---]разработать клиентскую часть веб-приложения;
    \item[---]провести различные виды тестирования веб-приложения.
\end{itemize}

\section{Требования к разрабатываемой системе}

Данный раздел описывает ключевые требования к проектируемой системе. Эти требования определяют ожидаемую функциональность и качественные характеристики системы, служа основой для дальнейшего проектирования и реализации. 

\subsection{Функциональные требования}

Платформа должна предоставлять пользователям необходимые инструменты для работы с учетными записями и вычислительными задачами. Пользователи должны иметь возможность самостоятельно регистрироваться в системе, входить под своей учетной записью и восстанавливать доступ при утере пароля. Также необходим просмотр базовой информации своего профиля и членства в группах.

Центральной функцией системы является управление контейнерами с исполяемыми приложениями.
Пользователи должны определять параметры задачи: имя, образ среды выполнения, команду запуска и переменные окружения.
Система должна поддерживать разные типы запуска: однократный, периодический по расписанию, с возможностью настройки этого расписания или по внешнему сетевому вызову.
Задачу можно создать, как в пространстве пользователя, так и при необходимости связать с группой для совместной работы.

Для контроля за задачами пользователи должны иметь возможность просматривать список доступных им задач и их текущее состояние, также получать детальную информацию о конфигурации и истории запусков конкретной задачи.
Важно обеспечить доступ к результатам выполнения каждого запуска, включая статус, логи и код завершения.
Платформа должна позволять перезапускать существующие задачи, отменять активные запуски и удалять задачи, прекращая их дальнейшее выполнение.
Для webhook-задач требуется механизм их активации по уникальному URL.

Система должна поддерживать создание групп для совместного использования ресурсов и управления задачами.
Пользователи должны иметь возможность присоединяться к группам и покидать их.
Администраторы групп должны иметь функционал исключения участников и управления правами доступа внутри группы (просмотр, редактирование, запуск, администрирование).
Все участники могут просматривать состав группы и права доступа, а администраторы — управлять процессом приглашения.

Необходимо реализовать логирование ключевых событий и ошибок для диагностики и анализа, а также предоставлять пользователям доступ к логам их задач.

\subsection{Нефункциональные требования}

Для безопасного взаимодействия с ситемой, должна быть разработана система аутентификации пользователей и контроля доступа на основе ролей. Среды выполнения пользовательских задач должны быть надежно изолированы друг от друга и от системных компонентов. Необходимо обеспечить защиту чувствительных данных при хранении и передаче информации, а также защиту от стандартных сетевых угроз.

Производительность системы должна обеспечивать быстрый отклик интерфейсов при ожидаемой нагрузке и эффективное использование вычислительных ресурсов. Масштабируемость архитектуры важна для обработки растущего числа пользователей и задач путем добавления ресурсов.

Удобство использования системы достигается через интуитивно понятный пользовательский интерфейс и логичный, хорошо документированный API.

Поддерживаемость и расширяемость должны обеспечиваться качественной архитектурой и кодом, что упрощает внесение изменений и добавление функционала.

\subsection{Требования к данным}

Система оперирует несколькими основными категориями данных, которые необходимо корректно хранить и обрабатывать.
К ним относятся данные пользователей, данные групп, данные задач и данные результатов выполнения.
Также используются справочные данные.
Важно обеспечить логическую целостность и непротиворечивость этих данных в хранилище.

\subsection{Требования к интерфейсам}

Основным интерфейсом для пользователей является графический пользовательский интерфейс, представляющий веб-приложение.
Для автоматизации и интеграции предоставляется программный интерфейс (API), построенный на стандартных веб-технологиях.

Предоставляемые интерфейсы должны быть спроектированы с учетом удобства пользователя (UI/UX).
Графический интерфейс должен быть интуитивно понятным, логически структурированным и визуально последовательным. Программный интерфейс (API) должен быть четко определен, логичен и удобен для интеграции, следуя современным практикам проектирования API.

\section{Вывод}


В первой главе дипломной работы был проведен анализ предметной области управления бессерверными вычислениями.
Были рассмыотрены существующие подходыы к кправлению вычисленияю, проанализированы существующие крупные решения, их преимущества и ограницения.

В ходе проведения анализа было вявлено что существующие решения в обласи FaaS и BaaS накладывают существенные ограничения на пользователей, делая неаозможными выполнение большого спектра задач.

Был проведен анализ существующих техничеких решений с открытым исходным кодом позволяющих добиться схожих возможностей управления бессерверными вычислениями на инфраструктуре пользователя. Было выявлено что такие решения требуют от пользователя повышенных затрат и компетенций для поддержания собственной инфраструктуры.

Целевая аудитория была разделена на неасколько сегментов, в том числе малые и средние предприятия, образовательные учереждения.
Анализ потребностей, рыночных трендов и конкурентного окружения позволил выделить ключевые требования целевой аудитории. На основе проведенного анализа сформулированы выводы и рекомендации к разработке.

В результате была сформирована цель разработки и сформулированы задачи необходимые для ее выполнения, а так же описаны функциональные и нефункциональные требования к разрабатываемой системе.

