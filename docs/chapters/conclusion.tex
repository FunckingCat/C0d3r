\newpage
\begin{center}
  \textbf{\large ЗАКЛЮЧЕНИЕ}
\end{center}
\refstepcounter{chapter}
\addcontentsline{toc}{chapter}{ЗАКЛЮЧЕНИЕ}

В ходе выполнения данной выпускной квалификационной работы ожидаемые результаты были достигнуты. Был разработан и реализован действующий прототип автоматизированной платформы для развертывания контейнеризованных функций в среде Kubernetes. Созданная платформа способна существенно упростить процессы разработки, тестирования и эксплуатации приложений, предоставляя разработчикам управляемую, безопасную и масштабируемую среду для выполнения их вычислительных задач.

В процессе работы над платформой были реализованы ключевые функциональные возможности, такие как управление жизненным циклом задач, централизованная аутентификация и авторизация пользователей с использованием Keycloak и гранулярной ролевой модели на основе групп.

Реализованный функционал позволяет достичь поставленных целей по автоматизации развертывания, обеспечению изоляции выполнения пользовательского кода и контролю доступа к ресурсам.

В процессе работы были решены следующие основные задачи:

\begin{itemize}
\item[---]проведено исследование предметной области, включая технологии контейнерной оркестрации, а также существующие подходы к реализации платформ FaaS и;автоматизации развертывания.
\item[---]осуществлен анализ требований к разрабатываемой платформе, определены ключевые сущности, пользовательские роли и основные сценарии использования;
\item[---]спроектирована архитектура системы, выбран технологический стек, определены компоненты системы и их взаимодействие, разработана спецификация API и;схема базы данных.
\item[---]разработана серверная часть и клиентская части платформы;
\item[---]настроена среда развертывания в Kubernetes, подготовлены необходимые манифесты и конфигурационные файлы;
\item[---]проведено функциональное тестирование разработанного прототипа для проверки работоспособности основных функций.
\end{itemize}

Тем не менее, работа над платформой не завершена, и существует потенциал для ее дальнейшего развития. На следующих этапах необходимо расширение функциональности, внедрение модульного и интеграционного тестирования, реализацию более гибких механизмов управления задачами, интеграция с системами непрерывной интеграции и доставки (CI/CD), внедрение механизмов квотирования ресурсов и детального мониторинга.

Исходя из вышеизложенного, можно заключить, что цель данной выпускной квалификационной работы — разработка прототипа автоматизированной платформы развертывания контейнеризованных функций — достигнута, а поставленные задачи успешно решены на текущем этапе, создав основу для дальнейшего развития продукта.